\section{Challenges}
\label{section:challenges}
\textbf{Autonomous Navigation}

One of the most difficult problems for autonomous navigation is that it must explore and keep track of the environment in real-time. Most importantly, we must make sure that autonomous navigation rarely makes mistakes since accidents might occur if any part of the hardware or software system encounters a single error. Perhaps a single parameter that was overweighted in a system may cause serious accidents.

Moreover, some researchers also debated that the system would face a dilemma when it encounters some ethical or psychological problems such as the Trolley Problem. That is, even if the autonomous navigation system is functional and did help the driver to save plenty of effort, it is still now served as a role of auxiliary instead of the driver.
\\\\
\textbf{Metaverse}

Metaverse has several challenges to overcome \cite{metaverseChallenge}, such as how to identify who the person is (which may further cause security problem) or lack of clarifying laws to protect and follow.

First, it is extremely essential to figure out how to prove who you are in virtual reality, instead of another person or even a bot trying to mimic your existence. This will require building new methods of personal data and privacy protection that will be able to ensure the safety of one’s identity and possessions in the virtual world. In other words, personal verification might come to the point where users will have to provide more personal data than what is expected today, to identify themselves and ensure the security system works efficiently, keeping personal data safe.

Finally, Metaverse is bound to bring large numbers of users together, making it a place of great opportunity to connect and exchange, however at the same time making users vulnerable in case there are no laws that regulate the boundaries. It will be a true challenge to identify jurisdiction as well as set legislation that can ensure the virtual space is safe and secure for its users.